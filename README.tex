% Options for packages loaded elsewhere
\PassOptionsToPackage{unicode}{hyperref}
\PassOptionsToPackage{hyphens}{url}
%
\documentclass[
]{article}
\usepackage{amsmath,amssymb}
\usepackage{lmodern}
\usepackage{iftex}
\ifPDFTeX
  \usepackage[T1]{fontenc}
  \usepackage[utf8]{inputenc}
  \usepackage{textcomp} % provide euro and other symbols
\else % if luatex or xetex
  \usepackage{unicode-math}
  \defaultfontfeatures{Scale=MatchLowercase}
  \defaultfontfeatures[\rmfamily]{Ligatures=TeX,Scale=1}
\fi
% Use upquote if available, for straight quotes in verbatim environments
\IfFileExists{upquote.sty}{\usepackage{upquote}}{}
\IfFileExists{microtype.sty}{% use microtype if available
  \usepackage[]{microtype}
  \UseMicrotypeSet[protrusion]{basicmath} % disable protrusion for tt fonts
}{}
\makeatletter
\@ifundefined{KOMAClassName}{% if non-KOMA class
  \IfFileExists{parskip.sty}{%
    \usepackage{parskip}
  }{% else
    \setlength{\parindent}{0pt}
    \setlength{\parskip}{6pt plus 2pt minus 1pt}}
}{% if KOMA class
  \KOMAoptions{parskip=half}}
\makeatother
\usepackage{xcolor}
\usepackage[margin=1in]{geometry}
\usepackage{graphicx}
\makeatletter
\def\maxwidth{\ifdim\Gin@nat@width>\linewidth\linewidth\else\Gin@nat@width\fi}
\def\maxheight{\ifdim\Gin@nat@height>\textheight\textheight\else\Gin@nat@height\fi}
\makeatother
% Scale images if necessary, so that they will not overflow the page
% margins by default, and it is still possible to overwrite the defaults
% using explicit options in \includegraphics[width, height, ...]{}
\setkeys{Gin}{width=\maxwidth,height=\maxheight,keepaspectratio}
% Set default figure placement to htbp
\makeatletter
\def\fps@figure{htbp}
\makeatother
\setlength{\emergencystretch}{3em} % prevent overfull lines
\providecommand{\tightlist}{%
  \setlength{\itemsep}{0pt}\setlength{\parskip}{0pt}}
\setcounter{secnumdepth}{-\maxdimen} % remove section numbering
\ifLuaTeX
  \usepackage{selnolig}  % disable illegal ligatures
\fi
\IfFileExists{bookmark.sty}{\usepackage{bookmark}}{\usepackage{hyperref}}
\IfFileExists{xurl.sty}{\usepackage{xurl}}{} % add URL line breaks if available
\urlstyle{same} % disable monospaced font for URLs
\hypersetup{
  pdftitle={Plasticity of the gastrocnemius elastic system in response to decreased work and power demand during growth},
  pdfauthor={Zanne Cox},
  hidelinks,
  pdfcreator={LaTeX via pandoc}}

\title{Plasticity of the gastrocnemius elastic system in response to
decreased work and power demand during growth}
\author{Zanne Cox}
\date{2023-01-29}

\begin{document}
\maketitle

\hypertarget{this-readme-details-the-data-analysis-files-for-this-project}{%
\subsection{This README details the data analysis files for this
project}\label{this-readme-details-the-data-analysis-files-for-this-project}}

\hypertarget{abstract}{%
\paragraph{Abstract}\label{abstract}}

Elastic energy storage and release can enhance performance that would
otherwise be limited by the force--velocity constraints of muscle.
Although functional influence of a biological spring depends on tuning
between components of an elastic system (the muscle, spring-driven mass
and lever system), we do not know whether elastic systems systematically
adapt to functional demand. To test whether altering work and power
generation during maturation alters the morphology of an elastic system,
we prevented growing guinea fowl (Numida meleagris) from jumping. We
compared the jump performance of our treatment group at maturity with
that of controls and measured the morphology of the gastrocnemius
elastic system. We found that restricted birds jumped with lower jump
power and work, yet there were no significant between-group differences
in the components of the elastic system. Further, subject-specific
models revealed no difference in energy storage capacity between groups,
though energy storage was most sensitive to variations in muscle
properties (most significantly operating length and least dependent on
tendon stiffness). We conclude that the gastrocnemius elastic system in
the guinea fowl displays little to no plastic response to decreased
demand during growth and hypothesize that neural plasticity may explain
performance variation

\hypertarget{data-analysis-pipeline}{%
\subsubsection{Data Analysis pipeline}\label{data-analysis-pipeline}}

\begin{enumerate}
\def\labelenumi{\arabic{enumi}.}
\item
  \textbf{Collect and Analyze Force data from jumping tests of 16 Guinea
  Fowl}

  \begin{itemize}
  \tightlist
  \item
    As described previously (Cox et al., 2020), at skeletal maturity
    (between 29 and 31 weeks old) jump performance was measured by
    placing each bird in turn on 6×6 inch (15.24×15.24 cm) force plates
    (AMTI HE6x6; Watertown, MA, USA) enclosed in a tapered box and
    encouraging the birds to jump
  \item
    Calculated:
  \end{itemize}

  \begin{enumerate}
  \def\labelenumii{\arabic{enumii}.}
  \tightlist
  \item
    Jump power from instantaneous net vertical ground reaction and the
    vertical center of mass velocity
  \item
    Instantaneous Velocity: integrating center of mass acceleration
  \item
    Jump work: integrating power with respec to time
  \end{enumerate}
\item
  \textbf{Build subject Specific Musculoskeletal models of individuals}

  \begin{itemize}
  \item
    Collect Morphology Measurements

    \begin{enumerate}
    \def\labelenumii{\arabic{enumii}.}
    \tightlist
    \item
      Muscle Analysis of lateral and medial gastroc
    \item
      Moment Arm of Achillies about the ankle
    \item
      Tendon force/Length
    \item
      Tendon Slack Length
    \item
      Leg bone segment lengths
    \end{enumerate}
  \item
    Fit model parameters to match experimentally measured values via
    particle swarm optimization
  \end{itemize}
\item
  \textbf{Simulate muscle activation and tendon strain across a range of
  postures}

  \begin{itemize}
  \tightlist
  \item
    Extract Elastic Energy Storage
  \end{itemize}
\item
  \textbf{Statistical Tests}
\item
  \emph{Do components of the gastrocnemius elastic system change
  systematically in response to changes in power and work demand during
  growth?}

  \begin{itemize}
  \tightlist
  \item
    To determine whether components of the gastrocnemius elastic system
    change systematically in response to changes in demand, we evaluated
    the influence of treatment group (restricted versus control) on each
    element of morphology measured. This was accomplished using t-tests
    if the homogeneity of variance assumption test was passed, and using
    a Kruskal--Wallis test by ranks when this criterion was not met. The
    threshold for statistical significance was set at 0.005 after a
    Bonferroni correction for multiple comparisons.
  \end{itemize}
\item
  \emph{Is the energy storage capacity reduced in individuals that did
  not jump during growth?}

  \begin{itemize}
  \tightlist
  \item
    The relationship between treatment group and elastic energy storage
    capacity was evaluated with a t-test after data passed tests for
    normality and homogeneity of variance, as described above for
    evaluation of differences between groups of individual elastic
    system components
  \end{itemize}
\item
  \emph{Which type of morphological variation has the greatest influence
  on energy storage capacity?}

  \begin{itemize}
  \tightlist
  \item
    We used stepwise comparison of Akaike information criterion (AIC)
    values (stepAIC R Mass package; Venables and Ripley, 2002) to
    determine the parameters and coefficients of the full model that
    best predicted elastic energy storage potential across natural
    variation of joint postures in preparation for jumps. The full
    statistical model evaluated included stored strain energy (PE) as a
    dependent factor and, as potential independent variables, tendon
    stiffness (tendonK), the summed maximum isometric force capacity of
    LG and MG along the tendon (sumFMax), the average LG and MG optimal
    fascicle length (avOFL) and starting muscle length. We included
    possible interaction terms between muscle force capacity, tendon
    stiffness and muscle start length (sumMaxF×tendonK×avLenA0c) and
    between optimal fascicle length and tendon stiffness (avOFL×tendonK)
    following recommendations by Zajac (1989) of functional equivalent
    muscle tendon joint properties at zero activation of the LG and MG
    in the pre-jump posture (avLenA0c).
  \end{itemize}
\item
  \emph{Does elastic energy storage capacity predict peak jump powers
  and work?}

  \begin{itemize}
  \tightlist
  \item
    The relationship between Achilles tendon elastic energy storage
    capacity and experimentally measured muscle-mass-normalized peak
    power output and jump work were both tested with a linear model with
    elastic energy storage as the dependent variable and peak power or
    jump work as the independent variable.
  \end{itemize}
\end{enumerate}

\end{document}
